%! TEX root = master.tex

\lecture{3}{Tue 26 Oct 2021 12:39}{Rank of a Matrix}

\definition{The rank of a matrix, denoted by rank$(\matr A)$, is equal to the number of non-zero rows in a row echelon form of \matri{A}}

\theorem{Let $\matr{AX} = \matr B$ be a linear system of $m$ linear equations in $n$ unknowns with augmented matrix $\left(\begin{array}{c:c}A & B\end{array}\right)$, then 

\begin{itemize}
  \item the system has a solution if and only if rank(\matri A) $=$ rank$\left(\begin{array}{c:c}A & B\end{array}\right)$
  \item the system has a uniue solution if and only if rank($\matr{A}$) $=$ rank$\left(\begin{array}{c:c}A & B\end{array}\right) = n$
  \end{itemize}
}
\note{A rank of a matrix augmented with another cannot be smaller than the original non-augmented matrix and thus }

\eg{
  For which values of $a$ does the following system have a unique solution? For which pairs of $a,b$ does the system have more than one solution?
    \begin{align*}
      x-2y &= 1\\
      x-y + az & = 2\\
      ay + 9z &= b
    \end{align*}
  \[\begin{pmatrix}[ccc:c] 
        1 & -2 & 0 & 1 \\
        1 & -1 & a & 2\\
        0 & a & 9 & b
    \end{pmatrix} 
   \mathrel{\underset{\sim}{R}} 
   \begin{pmatrix}[ccc:c] 
        1 & -2 & 0 & 1\\
        0 & 1 & 9 & 1\\
        0 & 0 & 9-a^2 & b-a
   \end{pmatrix}\]
 }

   The solutions is unique if and only if $9-a^2 \neq 0$ which is equivalent to $a \neq \pm 3 \implies$ rank($\matr A$) $=3$. The solutions are infinite if and only if $a = \pm 3$ and $b-a = 0$.

   \[(a,b) = (\pm 3, \pm 3) \implies \text{rank} \matr A) = 3\]

   If a system of $n$ equations in $n$ unknowns (\emph{also referred to as a square system}) has a unique solution, then the solution can be foudn by using the inverse of the coefficient matrix.

    \[\left.\begin{array}{r}
       AX=B\\
       A \text{ is } n \times n \\
       \text{The system has 1 solution}  
   \end{array}\right\} \matr X = \matr A^{-1} \matr B \]

   \note{
     An $n \times n$ matrix $\matr A$ is invertible if and only if rank($\matr A\text{)} = n$.
   }
