%! TEX root = master.tex

\lecture{6}{Tue 09 Nov 2021 14:59}{Eigenvalues}

\begin{definition}
  Let \matri{A} be an $n \times n$ matrix. A non-zero n-vector $\matr x$ such that
  $$\matr A \matr x = \lambda \matr x$$ is called an \textbf{eigenvector} of \matri A
  with corresponding \textbf{eigenvalue} $\lambda$.
\end{definition}~\\
To find the eigenvalues, consider $\matr{Ax} = \lambda \matr{x}$. After applying some
algebraic manipulations, particularly the factorisation of \matri x, which involves
the multiplication of $\lambda$ and the identity matrix, we get:
 \[ (\matr{A} - \lambda\matr{I})\matr{x} = 0\]
Above equation is a homogenous system of equations which must ahve a solution other
than $\matr x = 0$. 
The eigenvalues of any $n \times n$ matrix \matri{A} are the values of $\lambda$ that
satisfy the following equation, which is referred to as the \emph{characteristic
equation}. 
\[\det(\matr{A} - \lambda\matr I) = 0.\]
\begin{note}
  For an $n \times n$ matrix, the above equation turns out to be just a polynomial of
  degree $n$ in terms of $\lambda$. As expected, its roots might be real or complex
  as well as distinct or equal.
\end{note}
