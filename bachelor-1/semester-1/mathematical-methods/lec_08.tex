%! TEX root = master.tex

\lecture{8}{Tue 16 Nov 2021 14:29}{Diagonalization}

  An $n \times n$ matrix \matri A is diagonalisable if it is ``similar'' to a
  diagonal matrix \matri D, that is, there exists an invertible matrix \matri P such
  that $\matr D = \matr P \inverse \matr{AP}$. 

  In this case, \matri D is the diagonal
  matrices whose entries on the main diagonal are the eigenvalues $
  \lambda_1,\, \lambda_2,\, \cdots,\,\lambda_n$ of the initial matrix \matri{A}. \matri{P} is the square matrix whose
  column vectors are the eigenvectors of the initial matrix \matri{A}.

  \begin{definition}
    An $n \times n$ matrix \matri A is diagonalisable if the number of
    fundamental eigenvectors is equal to its dimension.
  \end{definition}

  Recall $\matr A = \left(\begin{smallmatrix} 3 & 2\\-1 & 0\end{smallmatrix}\right)$, whose eigenvalues are
  $\lambda_1 = 1,\,\lambda_2 = 2$; and whose fundamental eigenvectors are
  $\left(\begin{smallmatrix} -1\\1
      \end{smallmatrix}\right),\,\left(\begin{smallmatrix}
  2\\-1\end{smallmatrix}\right)$.
 
  \textbf{Ex.} 
  Thus, 
  \begin{equation*}
    \begin{aligned}
      \matr D &= \begin{pmatrix} 1 & 0\\ 0 & 2 \end{pmatrix}
    \end{aligned}
    \qquad
    \begin{aligned}
    \matr P &= \begin{pmatrix} -1 & 2\\1 & -1 \end{pmatrix}\\
    \matr P \inverse &= \begin{pmatrix} 1 & 2\\1 & 1 \end{pmatrix}
    \end{aligned}
  \end{equation*}

  \begin{problem}
    Find $\matr A^{5}$, where $\matr A = \begin{pmatrix} 3&2\\-1&0 \end{pmatrix}$.
    But $\matr D  = \matr P \inverse \matr{AP} \implies \matr A = \matr{PD} \matr P
    \inverse$. Then, 
\begin{align*}
  \matr A^{5} &= (\matr{PD} \matr P \inverse)^5\\
              &= \matr{PD} \matr P \inverse \cdot \matr{PD} \matr P \inverse \cdot
              \matr{PD} \matr P \inverse \cdot \matr{PD} \matr P \inverse \cdot
              \matr{PD} \matr P \inverse\\
              &=\matr P \matr D^5 \matr P \inverse
\end{align*}
  \end{problem}

  \begin{note}
    If \matri A is diagonalisable, then $\matr A^n = \matr P \matr D^n \matr P
    \inverse$
  \end{note}
