%! TEX root = master.te

\lecture{13}{Wed 08 Dec 2021 15:34}{Differentials and Exact Equations}

\begin{definition}
For a differentiable function of two variables given by $f(x,y)$ we define the
differentials of $f$, $dx$ and $dy$, to be two independent variables. Then, the
differentiable $df$, also called the total differential is defined as 
\[df =  \p F x dx + \p F ydy.\]
\end{definition}

\begin{example}
Find the differential of the function $z=x^3\ln(y^2)$.
\end{example}

Consider the following level curves of a function of two variables $F(x,y) = C$,
where $C$ is a constant in the range of $F$. On each curve, the total differential of
$F$ along the curve is zero, namely:
\[dF(x,y) =  p F x dx +  \p F y dy  = 0.\]
Dividing by $dx$ and arranging the terms we get 
\[\frac{dy}{dx} = \p F x \cdot \p y F\]
Whilst being a function in $x$ and $y$, the above equaion is also a first order
differential equation. Reversing the process, we can solve any give differential
equation, since any first order differential equation can be written as 
\[M(x,y)dx + N(x,y)dy = 0.\]
The LHS can be then written as the total differential equation of some function $F(x,y)$
as \[M(x,y) dx + N(x,y) dy =  dF(x,y)\] if there is a function $F(x,y)$ such that
$\p F x= M(x,y)$ and $\p F y = N(x,y)$. Thus the
general solution of an exact equation is given by the level curve $F(x,y) = C$.

\begin{example}
	Solve $\underbrace{\frac{2}{\sqrt{1-x^2}} + y\cos(xy)}_{\frac{\partial F}{\partial x}}dx 
		  + \underbrace{x\cos(xy) - y^{-\frac{1}{3}}}_{\frac{\partial F}{\partial y}}dy = 0$
\end{example}
Considering $\p{F}{x}$ and $\p{F}{y}$ seperately we get
\begin{align*}
	\p F x  &= \int \frac{2}{\sqrt{1-x^2}} + y\cos(xy)\,dx = 2\arcsin(x) + \sin(xy)\,dx\\
	\p F y &= \int x\cos(x) - y^{-\frac{1}{3}} = \sin(xy) - \frac{3y^{\frac{2}{3}}}{2}\,dy
\end{align*}
so taking lowest common sum of the terms for each integral we accomodate our general solution as follows:
\[2\arcsin(x) + \sin(xy) - \frac{3}{2}y^{\frac{2}{3}} = C \qed\]
\begin{example}
	Find the most general function $N(x, y)$ such that the following is an exact equation:
	$(ye^{xy}-4x^3y+2)dx + N(x, y)dy = 0$.
\end{example}
