%! TEX root = master.tex

\lecture{2}{Mon 25 Oct 2021 14:05}{Floating point addition}

Converting $2.25_{10}$ into \textbf{IEEE 754}:

\begin{align*}
  2.25 &= 2 + 0.25\\
       &= 2^1 + 2^{-2}\\
       &= (1 + 2^{-3}) \times 2^1\\
       &= (-1)^s \times (1 + m) \times 2^{(x-127)} \\
       &= 0\ 10000000\ 0010\ 000\ 000\ 000
\end{align*}
\begin{center}
\begin{tabular}{c|c|c}
 $s$ & $x$ & $m$       \\ \hline
1        & 1000 0000           & 0010 0000 0000 000 \\
\end{tabular}
\end{center}

Converting $134.0625_{10}$ into \textbf{IEEE 754}:

\begin{align*}
  134.0625 &= \left(\frac{2^7}{2^7} + \frac{6}{2^7} + \frac{0.0625}{2^7}\right) \times 2^7\\
           &= (1 + \frac{6}{2^{7}} + \frac{2^{-4}}{2^7})
\end{align*}

In floating point arithmetic, operations become more complex. The following is the algorithm for addition:

\begin{itemize}
  \item Check for zeros. If one operand is zero, the result is the other operand.
  \item Align the signifcands of the other operands.
  \item Add the significands.
  \item Normalize the result.
\end{itemize}



