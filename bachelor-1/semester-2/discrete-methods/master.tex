\documentclass[a4paper]{article}
% Some basic packages
\usepackage[utf8]{inputenc}
\usepackage[T1]{fontenc}
\usepackage[a4paper, total={6in, 8in}]{geometry}
\usepackage{tikz}
\usepackage{textcomp}
\usepackage{url}
\usepackage{graphicx}
\usepackage{float}
\usepackage{booktabs}
\usepackage{enumitem}
\usepackage{arydshln}
\delimitershortfall=0pt
\setlength{\dashlinegap}{2pt}

\usepackage{amsmath,amssymb,fancyhdr,array}

\pagestyle{fancy}

\usepackage{xintfrac, xintbinhex, xinttools}

\rhead{1608/1609 WS18/19}

\makeatletter
\def\DtoB@get@ND #1/#2[#3]{% #2 must be = 1 here, if input was in
                         % decimal notation
    \edef\DtoB@s{\xintiiSgn{#1}}%
    \edef\DtoB@A{\xintiiAbs{#1}}%
    \def\DtoB@L{#3}%
    \ifnum#3<\z@
       \let\DtoB@N\DtoB@A
       \edef\DtoB@D{\xintiiE{1}{-#3}}%
    \else
       \edef\DtoB@N{\xintiiE{\DtoB@A}{#3}}%
       \def\DtoB@D{1}%
    \fi
}%
\newcommand\ParseFromDecimalToIEEEBinary[1]{%
   % assume #1 is a decimal number (I will write code for fractions
   % another day)
   \expandafter\DtoB@get@ND\romannumeral0\xintrez{#1}%
   % we should here handle \DtoB@s = 0 but no time tonight, assume input non-zero
   \edef\DtoB@S@bit{\the\numexpr(1-\DtoB@s)/2}% 1 if number < 0, 0 if number > 0
   \edef\DtoB@U{\xintDecToBin{\DtoB@N}}%
   \edef\DtoB@V{\xintDecToBin{\DtoB@D}}%
   \edef\DtoB@Uk{\the\numexpr\expandafter\xintLength\expandafter{\DtoB@U}-\@ne}%
   \edef\DtoB@Vl{\the\numexpr\expandafter\xintLength\expandafter{\DtoB@V}-\@ne}%
   % next step should perhaps compare k and l first
   % important that we are comparing here two strings of 1s and 0s of
   % exact same length
   \ifnum\pdfstrcmp{\DtoB@U\romannumeral\xintreplicate{\DtoB@Vl}{0}}
                   {\DtoB@V\romannumeral\xintreplicate{\DtoB@Uk}{0}}=\m@ne
       \edef\DtoB@E{\the\numexpr\DtoB@Uk-\DtoB@Vl-\@ne}%
   \else
       \edef\DtoB@E{\the\numexpr\DtoB@Uk-\DtoB@Vl}%
   \fi
   \edef\DtoB@Eshifted@bits{\expandafter\@gobble
          \romannumeral0\xintdectobin{\the\numexpr \DtoB@E + 127 + 256\relax}}%
   \ifnum\DtoB@E>23
     \edef\DtoB@f@frac{\DtoB@A/\xintiiPow{2}{\DtoB@E-23}[\DtoB@L]}%
     %  use rather bintodec conversion of 10000...000 in above?
   \else
     \edef\DtoB@f@frac{\xintiiMul{\DtoB@A}{\xintiiPow{2}{23-\DtoB@E}}/1[\DtoB@L]}%
   \fi
   \edef\DtoB@f@int{\xintNum{\DtoB@f@frac}}% truncates to an int
   \edef\DtoB@M@bits{\expandafter\@gobble
                     \romannumeral0\xintdectobin{\DtoB@f@int}}%
   %\edef\DtoBresult{\DtoB@S@bit\DtoB@Eshifted@bits\DtoB@M@bits}%
   \let\IEEEsign\DtoB@S@bit
   \let\IEEEexponent\DtoB@Eshifted@bits
   \let\IEEEmantissa\DtoB@M@bits   
}%

\makeatother

\newcommand\ieee[1]{%
\ParseFromDecimalToIEEEBinary{#1}%
\begin{center}
{\footnotesize\setlength{\tabcolsep}{1pt}\begin{tabular}[t]{|*{32}{w{c}{1em}|}}
\firsthline
\xintListWithSep{&}{\xintSeq[-1]{31}{0}}\\
\hline
S \romannumeral\xintreplicate{8}{&E}\romannumeral\xintreplicate{23}{&M}\\
\hline
\IEEEsign & \xintListWithSep{&}{\IEEEexponent} & \xintListWithSep{&}{\IEEEmantissa}\\
\hline
\end{tabular}\par}%
\end{center}
}



% Don't indent paragraphs, leave some space between them
\usepackage{parskip}

\usepackage{subcaption}
\usepackage{multicol}
\usepackage{xcolor}

% Other font I sometimes use.
% \usepackage{cmbright}

% Math stuff
\usepackage{amsmath, amsfonts, mathtools, amsthm, amssymb, braket}
% Fancy script capitals
\usepackage{mathrsfs}
\usepackage{cancel}

% Bold math
\usepackage{bm}

%Make implies and impliedby shorter
\let\implies\Rightarrow
\let\impliedby\Leftarrow
\let\iff\Leftrightarrow
\let\epsilon\varepsilon

% Add \contra symbol to denote contradiction
\usepackage{stmaryrd} % for \lightning
\newcommand\contra{\scalebox{1.5}{$\lightning$}}

% \let\phi\varphi

% Command for short corrections
% Usage: 1+1=\correct{3}{2}

\definecolor{correct}{HTML}{009900}
\newcommand\correct[2]{\ensuremath{\:}{\color{red}{#1}}\ensuremath{\to }{\color{correct}{#2}}\ensuremath{\:}}
\newcommand\green[1]{{\color{correct}{#1}}}

% horizontal rule
\newcommand\hr{
    \noindent\rule[0.5ex]{\linewidth}{0.5pt}
}

% hide parts
\newcommand\hide[1]{}

% Environments
\makeatother
% For box around Definition, Theorem, \ldots
\usepackage{mdframed}
\mdfsetup{skipabove=1em,skipbelow=0em}
\theoremstyle{definition}
\newmdtheoremenv[nobreak=true]{definition}{Definition}
\newmdtheoremenv[nobreak=true]{property}{Property}
\newmdtheoremenv[nobreak=true]{consequence}{Consequence}
\newmdtheoremenv[nobreak=true]{lemma}{Lemma}
\newmdtheoremenv[nobreak=true]{proposition}{Proposition}
\newmdtheoremenv[nobreak=true]{theorem}{Theorem}
\newmdtheoremenv[nobreak=true]{law}{Law}
\newmdtheoremenv[nobreak=true]{corollary}{Corollary}
\newmdtheoremenv{conclusion}{Conclusion}
\newmdtheoremenv{assumption}{Assumption}
\newtheorem*{intermezzo}{Intermezzo}
\newtheorem*{observation}{Observation}
\newtheorem*{exercise}{Exercise}
\newtheorem*{remark}{Remark}
\newtheorem*{problem}{Problem}
\newtheorem*{terminology}{Terminology}
\newtheorem*{application}{Application}
\newtheorem*{question}{Question}
\newtheorem*{example}{Example}
\newtheorem*{notation}{Notation}
\newtheorem*{previouslyseen}{As previously seen}

\newcommand{\matr}[1]{\mathbf{#1}} % undergraduate algebra version
\newcommand{\matri}[1]{$\mathbf{#1}$} % undergraduate algebra version
\renewcommand{\mod}[1]{\left|\matr{#1}\right|}
\newcommand{\rank}[1]{\text{rank}\left(\mathbf{#1}\right)}
\newcommand{\inverse}{^{-1}}
\newcommand{\augmented}[2]{\begin{array}{c:c}\mathbf{#1} & \mathbf{#2}\end{array}}
\newcommand{\rowequiv}{\mathrel{\underset{\sim}{R}}}
\newtheorem*{note}{Note}
\newmdtheoremenv[nobreak=true]{prop}{Proposition}
% End example and intermezzo environments with a small diamond (just like proof
% environments end with a small square)
\usepackage{etoolbox}
\AtEndEnvironment{vb}{\null\hfill$\diamond$}%
\AtEndEnvironment{intermezzo}{\null\hfill$\diamond$}%
% \AtEndEnvironment{opmerking}{\null\hfill$\diamond$}%

% Fix some spacing
% http://tex.stackexchange.com/questions/22119/how-can-i-change-the-spacing-before-theorems-with-amsthm
\makeatletter
\def\thm@space@setup{%
  \thm@preskip=\parskip \thm@postskip=0pt
}

  \renewcommand*\env@matrix[1][*\c@MaxMatrixCols c]{%
    \hskip -\arraycolsep
    \let\@ifnextchar\new@ifnextchar
  \array{#1}}


% Exercise 
% Usage:
% \oefening{5}
% \suboefening{1}
% \suboefening{2}
% \suboefening{3}
% gives
% Oefening 5
%   Oefening 5.1
%   Oefening 5.2
%   Oefening 5.3
\newcommand{\oefening}[1]{%
    \def\@oefening{#1}%
    \subsection*{Oefening #1}
}

\newcommand{\suboefening}[1]{%
    \subsubsection*{Oefening \@oefening.#1}
}

\newcommand{\prob}{\mathbb P}

% \lecture starts a new lecture (les in dutch)
%
% Usage:
% \lecture{1}{di 12 feb 2019 16:00}{Inleiding}
%
% This adds a section heading with the number / title of the lecture and a
% margin paragraph with the date.

% I use \dateparts here to hide the year (2019). This way, I can easily parse
% the date of each lecture unambiguously while still having a human-friendly
% short format printed to the pdf.

\usepackage{xifthen}
\def\testdateparts#1{\dateparts#1\relax}
\def\dateparts#1 #2 #3 #4 #5\relax{
    \marginpar{\small\textsf{\mbox{#1 #2 #3 #5}}}
}

\def\@lecture{}%
\newcommand{\lecture}[3]{
    \ifthenelse{\isempty{#3}}{%
        \def\@lecture{Lecture #1}%
    }{%
        \def\@lecture{Lecture #1: #3}%
    }%
    \subsection*{\@lecture}
	%No date 
  % \marginpar{\small\textsf{\mbox{#2}}}

}



% These are the fancy headers

% LE: left even
% RO: right odd
% CE, CO: center even, center odd
% My name for when I print my lecture notes to use for an open book exam.
% \fancyhead[LE,RO]{Gilles Castel}
\usepackage{fancyhdr}
\pagestyle{fancy}
\fancyhf{}

\fancyhead[RO,LE]{Giorgio Grigolo}
% Right odd,  Left even
\fancyfoot[LE, RO]{\thepage}
% Todonotes and inline notes in fancy boxes
\usepackage{todonotes}
\usepackage{tcolorbox}

% Make boxes breakable
\tcbuselibrary{breakable}



% Figure support as explained in my blog post.
\usepackage{import}
\usepackage{xifthen}
\usepackage{pdfpages}
\usepackage{transparent}
\newcommand{\incfig}[1]{%
    \def\svgwidth{\columnwidth}
    \import{./figures/}{#1.pdf_tex}
}
 %http://tex.stackexchange.com/questions/76273/multiple-pdfs-with-page-group-included-in-a-single-page-warning
\pdfsuppresswarningpagegroup=1




% My name
\author{Giorgio Grigolo}

\title{Discrete Methods}
\def\propositionautorefname{\textbf{Proposition}}
\begin{document}
	\maketitle
	\tableofcontents
	% start lectures
	% end lectures	
	\section{Tuples, Permutations Sets and Multisets}
	
	
	\begin{proposition}\label{prop:1.1}
		$\mod{X \times Y} = \mod X \cdot \mod Y $
	\end{proposition}
	
	\begin{proof}
		Let $x_1, x_2 \cdots x_m$ be the elements of $X$ and let $y_1, y_2, \cdots y_n$ be the elements of $Y$. $X \times Y$ consists of all the pairs with the choices for the first element from $X$ and the choices for the second element from $Y$. The number of elements of $X \times Y$ is the sum of $n$ $m$'s, i.e. $mn$.
	\end{proof}
	
	
	\begin{proposition}\label{prop:1.2}
		If $A_1, A_2, \cdots, A_k$ are mutually disjoint sets, then $\mod{A_1 \cup A_2 \cup \cdots \cup A_n} = \mod{A_1} + \mod{A_2} + \cdots + \mod{A_k}$, i.e.: \[\mod{\bigcup_{k\in\mathbb N}A_k} = \sum_{k\in\mathbb N} \mod{A_k}\]
	\end{proposition}
	
	\begin{proof}
		By induction on k. This proposition is a special case of the \textbf{Inclusion-Exclusion} principle where the unions of all the sets are the empty set and thus have size 0.
	\end{proof}
	
	\newpage
	
	\subsection{Repetition and Order}
	
	Consider the set $[5] = \set{1,2,3,4,5}$. Suppose we had to pick 4 numbers from this set, possibly with repetition. So the set of all choices would look like the following: \[[5]^4= \set{(1,1,1,1),\, (1,1,1,2),\, (1,1,2,1),\, \cdots,\, (5,5,5,5)}\] which is equivalent to $[5] \times [5] \times [5] \times [5]$. The following proposition will enlighten us on how to calculate the size of said set.
	
	\begin{proposition}\label{prop:1.3}
		For all $k \geq 2$ sets $X_1, X_2, \cdots, X_k$, \[\mod{X_1 \times X_2 \times \cdots \times X_k} = \mod{X_1} \times \mod{X_2} \times \cdots \times \mod{X_k}.\]
	\end{proposition}
	
	
	
	\begin{proof}
		By induction on $k$, the base case $k=2$ is given by \textbf{\autoref{prop:1.1}}. Now, consider $k\geq3$. Let $x_1, x_2,\cdots,x_k$ be the elements of $X_k$ and $A= X_1 \times X_2, \times \cdots \times X_k$, or rather \[A = \set{x_{11}, x_{12}, \cdots, x_{1k}} \times \set{x_{21}, x_{22}, \cdots, x_{2k}} \times \cdots \times \set{x_{k1}, x_{k2}, \cdots, x_{kk}}.\] For $j \in \mathbb N$, let $A_j$ be the set of $k$-tuples ending in $x_j\in X_1$. \marginpar{\protect\footnotesize\fbox{$A_j \subseteq A$.}} Therefore, a $k-tuple$ is in $A_j$ if and only if its first $k-1$ entries form a $(k-1)$-tuple in $X_1 \times X_2 \times \cdots \times X_{k-1}$ and its $k$-th entry is $x_j$. Thus, $\mod{A_j}$, or the number of $k$-tuples in $A_j$ is $\mod{X_1 \times X_2 \times \cdots \times X_{k-1}}$, which is $\mod{X_1} \times \mod{X_2} \times \cdots \times \mod{X_{k-1}}$ by the inductive hypothesis. 
		
		Next, we show that $A = \bigcup_{n\in\mathbb N}A_n$. Let $a\in A$. Then, the $k$-th entry of $a$ is $x_i$ for some $i \in [n]$. So $a\in A_i$ and hence $a \in \bigcup_{n\in\mathbb N}A_n$ and therefore $A \subseteq \bigcup_{n\in\mathbb N}$. Now let $b \in \bigcup_{n\in\mathbb N} A_n$. Then, $b$ is an element of at least one set of $A_1, A_2, \cdots, A_n$. Since $\forall j\in\mathbb N,\, A_j \subseteq A$, we have $b \in A$ and therefore $\bigcup_{n\in\mathbb N}A_n \subseteq A$. Finally, $A = \bigcup_{n\in\mathbb N} A_n$.
		
		Now, $A_1, A_2, \cdots, A_n$ are mutually disjoint, because for any two sets $A_i$ and $A_j$ with $i\neq j$, any $k$-tuple in $A_i$ ends with $x_i$, whilst any $k$-tuple in $A_j$ ends with $x_j$. This fact allows us to use \textbf{\autoref{prop:1.2}} as follows:
		
		\begin{align*}
			A &= A_1 \cup A_2 \cup \cdots \cup A_k\\
			\mod A & = \mod{A_1} + \mod{A_2} + \cdots + \mod{A_n}                                                   \\
			& = \underbrace{\mod{X_1} \times \mod{X_2} \times \cdots \times \mod{X_{k-1}}}_{n \text{ times}} \\
			& = \left(\mod{X_1} \times \mod{X_2} \times \cdots \times \mod{X_{k-1}}\right)n                  \\
			& = \mod{X_1} \times \mod{X_2} \times \cdots \times \mod{X_{k-1}} \times \mod{X_k}               
		\end{align*}\marginpar{\vspace{-3em}\protect{\footnotesize\fbox{$n=\mod{X_k}$}}}
	\end{proof}
	
	\begin{corollary}
		For any $k\geq 2$ and any set $X$, \[\mod{X^k} = \mod{X}^k.\]
	\end{corollary}
	
	\begin{proof}
		This is given by \textbf{\autoref{prop:1.3}} with $X_1 = X_2 = \cdots = X_k = X$
	\end{proof}
	
	\newpage
	
	\subsection{\cancel{Repetition} and Order}
	
	Suppose we had to choose $n$ objects from $k$ distinct objects, that each object can only be chosen once, and that the order in which we picked the $n$ things mattered. We can label the set of $n$ objects $\set{1,2,\cdots,n} = [n]$. Each unique possibility can be represented by a $k$-tuple $(a_1, a_2, \cdots, a_k)$, but this time we have $a_1 \neq a_2 \neq \cdots \neq a_k$, or rather, all $a_k$ are distinct elements from $[n]$. Let $P^k_n$ be the set of such $k$-tuples defined as follows:	\[P^k_n = \set{(a_1, a_2, \cdots, a_k) \in [n]^k : a_1 \neq a_2 \neq \cdots \neq a_k}.\] The problem is now reduced into finding the size of $P^k_n$, since any $k$-tuple in this set represents a valid possibility, and all possibilities must be in this form.
	
	\begin{note}
		Since we cannot choose more objects ($k$) than the amount we have ($n$) we have to assume $n$ is at least $k$. So we have: \[P^k_n = \emptyset \, \iff \, k \leq n.\]
	\end{note}
	
	For example, if we had to pick 3 objects from a pool of 4, so $n=4$ and $k=3$, then the possibilities would be:
	
	\begin{center}
		(1, 2, 3), (1, 3, 2), (2, 1, 3), (2, 3, 1), (3, 1, 2), (3, 2, 1),\\
		(1, 2, 4), (1, 4, 2), (2, 1, 4), (2, 4, 1), (4, 1, 2), (4, 2, 1),\\
		(1, 3, 4), (1, 4, 3), (3, 1, 4), (3, 4, 1), (4, 1, 3), (4, 3, 1),\\
		(2, 3, 4), (2, 4, 3), (3, 2, 4), (3, 4, 2), (4, 2, 3), (4, 3, 2).
	\end{center}

	\begin{recall}
		The product $n\times (n-1) \times (n-2) \cdots \times 2 \times 1$ is represented by $n!$. In general, $\prod_{i=1}^n a_i$ represents the product $a_1 \times a_2 \times \cdots \times a_n$. Thus, $n! = \prod_{i=1}^m i$. We define $0!$ as 1.
	\end{recall}`
	
	\begin{proposition}
		For $1 \leq k \leq n,$ \[\mod{P^k_n} = n \times (n-1) \times (n-2) \times \cdots \times (n-k+1) = \frac{n!}{(n-k)!}\]
	\end{proposition}
	
	\begin{proof}
		By induction on $k$. Consider the base case $k=1$. The set $P^1_n$ is $\set{(1),(2),\cdots (n)}$ and has size $n = \frac{n!}{(n-1)!}$ as required.
		
		Now consider $k \geq 2$. For each $i\in [n]$, let $A_i$ be the set of tuples in $P^k_n$ which end with $i$. Then, a tuple $(a_1, a_2, \cdots, a_k) \in P^k_n$ is in $A_i$ if and only if $a_k = i$ and $a_1, a_2, \cdots, a_{k-1}$ are distinct elements of $[n]\setminus\{i\}$. So, a tuple in $A_i$ would look like the following: \[A_i = \left\{\left(a_1,\, a_2,\, a_3,\, \cdots,\, a_{k-1}, i\right): a_1\neq a_2  \neq \cdots \neq a_{k-1} \wedge\ \forall j\in \mathbb N,\, a_j \in [n]\setminus\{i\}\right\}\] Thus, size of $A_i$ is the number of $(k-1)$-tuples such that $a_1,\,a_2,\,\cdots,\, a_{k-1}$ are distinct elements of $[n]\setminus\{i\}$. By applying the inductive hypothesis
		
		\begin{align*}
			\mod{A_i} &= \mod{P^{k-1}_{n-1}}\\
					  &= \frac{(n-1)!}{\left((n-1)-(k-1)\right)!} = \frac{(n-1)!}{(k-1)!}
		\end{align*}
		
		By an argument similar to that in \textbf{\autoref{prop:1.3}}, we have that $P^k_n = \bigcup_{n\in\mathbb N} A_n$ and that $A_1,\, A_2,\, \cdots,\, A_n$ are mutually disjoint. By \textbf{\autoref{prop:1.2}}, \[\mod{P^k_n}  = \mod{A_1} + \mod{A_2} + \cdots + \mod{A_n}\] as required.
	\end{proof}

	\begin{note}
		$P^n_n$ is the set of all permutations of the elements of set $[n]$. Therefore, the number of permutations of a set of $n$ objects is $\frac{n!}{(n-n)!} = n!$. We can refer to $P^k_n$ as the set of permutations of the $k$-element subsets of $[n]$.
	\end{note}
	
	\subsubsection{Stirling's Approximation}
	
	For two functions $f$ and $g$ we write that $f(n) \sim g(n)$ if \[\frac{f(n)}{g(n)} \to 1 \text{ as } n\to \infty\]
	
	A result using this notation is \textbf{Stirling's Approximation}, which approximates the value of $n!$, goes as follows: \[n! \sim \sqrt{2n\pi}\left(\frac{n}{e}\right)^n\]
	
	A direct consequence of \textbf{Stirling's Approximation} gives us this result, quotable without proof:
	 \[\left(\frac{n}{e}\right) \leq n! \leq \frac{(n+1)^{n+1}}{e^{n}}.\]

     \subsection{\cancel{Repetition}, \cancel{Order}}


\end{document}
